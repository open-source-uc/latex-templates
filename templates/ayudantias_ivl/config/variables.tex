% Definiciones de comandos, para reutilizar secuencias frecuentes o ahorrar código
\newcommand{\mytitle}{Ayudantía 00}
\newcommand{\tema}{Tema de la Ayudantía}
\newcommand{\fecha}{00-00-0000}

\newcommand{\ayudante}{Nombre Ayudante}
\newcommand{\mailuc}{mail@uc.cl}

\newcommand{\facultad}{Nombre Facultad}
\newcommand{\semestre}{Primer Semestre del 0000}

\newcommand{\siglacurso}{AAA0000}
\newcommand{\nombrecurso}{Nombre Curso}
\newcommand{\numseccion}{0}
\newcommand{\profesor}{Nombre del Profesor}

\pagestyle{fancy}
\fancyhf{}
\renewcommand{\headrulewidth}{0pt}
\renewcommand{\footrulewidth}{0.35pt}

% Ubicación de figuras
\graphicspath{{./figuras/}}

% Definir color de hipervínculos
\hypersetup{
  colorlinks=false,
  linkbordercolor=0.96 0.60 0.14,   % Código RGB
  urlbordercolor=0.96 0.60 0.14,    % Código RGB
% hidelinks                         % Descomentar esta línea borra los bordes alrededor de hipervínculos
}

% Creación del pie de página
\lfoot{\fecha \hfill \siglacurso \ -- \mytitle \hfill Página \thepage{} de \pageref{LastPage}}