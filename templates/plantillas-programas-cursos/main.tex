\documentclass[11pt,letterpaper]{article}

\usepackage{fullpage}
\usepackage{fancyhdr}
\usepackage{listings}
\usepackage{anysize}
\usepackage{lscape}
\usepackage{soul}
\usepackage{graphicx}
\usepackage{listings}
\usepackage{xcolor}
\usepackage{multicol}
\usepackage[spanish]{babel}
\selectlanguage{spanish}
\usepackage[utf8]{inputenc}
\usepackage{termcal}
\usepackage{xcolor}
\marginsize{2cm}{2cm}{2cm}{2cm}
\usepackage{anysize}

\begin{document}

\raisebox{-18pt}
{\includegraphics[scale=0.1]{./Img/logo_puc.pdf}}
\vbox{
  \hbox{\sc Pontificia Universidad Católica de Chile}
  \hbox{\sc Escuela de Ingeniería}
  \hbox{\sc Departamento de Ciencia de la Computación}
}

\begin{center}
  \vspace{2ex}\Large Programa del Curso \\
\end{center}

\begin{tabular}{ l c l}
  Curso      & : & Inteligencia competitiva y vigilancia tecnológica para promover la innovación \\
  Código     & : & INP3540 \\
  Carácter   & : & Optativo \\
  Semestre   & : & 1er semestre 2017 \\
  Requisitos & : & Autorización del profesor \\
  Profesor   & : & Elvira Saurina \\
             & : & Jocelyn Patterson \\
  %Asistente & : & - \\
  Horario    & : & Cátedra - Miércoles 18:30 a 21:20 hrs. \\
           % & : & Ayundatía - \\
  Sala       & : & H4 \\
\end{tabular}

\section*{Descripción}
En este curso, se entregarán los conocimientos e instrumentos característicos para desarrollar servicios de inteligencia competitiva (IC) y vigilancia tecnológica (VT) en ambientes orientados a promover actividades de I+D+i. La inteligencia competitiva y la vigilancia tecnológica, por medio de la captación, selección, análisis, difusión y comunicación de información relevante de la propia organización y del entorno científico-tecnológico, ayudan a generar nuevo conocimientos para tomar decisiones estratégicas con menor riesgo y anticiparse a los cambios. Se desarrollarán habilidades para la búsqueda, interpretación y difusión de información de utilidad para la toma de decisiones de una empresa.

\section*{Objetivos}
Al finalizar el curso, el estudiante deberá ser capaz de Diseñar, implementar y evaluar servicios de inteligencia competitiva y vigilancia tecnológica para fomentar la innovación en organizaciones.

\begin{itemize}
  \item Distinguir y aplicar los procesos principales de la IC/VT.

  \item Identificar necesidades de información en una organización.

  \item Seleccionar y usar las fuentes de información más apropiadas según tipo de necesidad o proyecto de investigación.

  \item Analizar, sintetizar y comunicar información y conocimientos en una amplia variedad de formatos.

  \item Analizar problemas de información y desarrollar soluciones, basándose en diversas herramientas tecnológicas y las mejores prácticas.
\end{itemize}

\section*{Contenidos}

\begin{enumerate}
  \item Introducción a la Inteligencia Competitiva (IC) y a la Vigilancia Tecnológica (VT).
    \begin{itemize}
      \item Definición y conceptos
      \item Enfoques de la Inteligencia Competitiva.
      \item Rol de la Innovación
    \end{itemize}

  \item Estudio de las necesidades de IC y VT
    \begin{itemize}
      \item  Definición de necesidades
      \item Técnicas, métodos e instrumentos
    \end{itemize}

  \item Fases del proceso de VT
    \begin{itemize}
      \item Círculo de Inteligencia
      \item Actividades IC/VT
    \end{itemize}

  \item Búsqueda de información
    \begin{itemize}
      \item Clasificación de las fuentes: primarias y secundarias
      \item Patentes y marcas: bases de datos y directorios
    \end{itemize}

  \item Análisis, tratamiento y difusión de la información
    \begin{itemize}
      \item Modelos de análisis
      \item Análisis de patentes y análisis bibliométricos
      \item Tratamiento de la información: rol de los expertos y validación de la información
      \item Difusión de la información: herramientas.
    \end{itemize}

  \item Diseño de un sistema de VT en una empresa
\end{enumerate}

\section*{Metodología}
El curso cuenta con la siguiente metodología:
\begin{itemize}
  \item Exposición del profesor
  \item Lecturas comentadas.
  \item Estudio de casos
  \item Presentaciones de los alumnos.
\end{itemize}

\section*{Evaluación}
\begin{itemize}
  \item Proyecto final: 30\% (entrega informe final y presentación 19 de julio). Entregas parciales: 15 mayo, 29 mayo y 3 julio.

  \item Controles de lectura (sorpresa): 20\%
  \item Análisis de casos y presentaciones de lecturas: 20\%
  \item Descripción de ámbito tecnológico: 10\% (presentación en clase el 21 de junio).
  \item Participación en clase: 20\%.
\end{itemize}

%\bibliographystyle{abbrv}
%\bibliography{Refs/bibliografiaINP3540}

\newpage

\section*{Programación semestral}

\begin{calendar}{3/20/2017}{18}
  \setlength{\calboxdepth}{.3in}

  % Description of the Week.
  \calday[Lunes]{\noclassday} % Sunday
  \calday[Martes]{\noclassday} % Tuesday
  \calday[Miércoles]{\classday} % Wesdnesday
  \calday[Jueves]{\noclassday} % Thursday
  \calday[Viernes]{\noclassday} % Friday
  \skipday\skipday % weeekend

  % Teaching assistance

  % Holidays
  \options{3/29/2017}{\noclassday}
  \caltext{3/29/2017}{\color{red}No hay clases}

  \options{4/13/2017}{\noclassday}
  \caltext{4/13/2017}{\color{red}No hay clases}

  \options{4/19/2017}{\noclassday}
  \caltext{4/19/2017}{\color{red}No hay clases}

  \options{5/1/2017}{\noclassday}
  \caltext{5/1/2017}{\color{red}No hay clases}

  \options{6/26/2017}{\noclassday}
  \caltext{6/26/2017}{\color{red}No hay clases}

  % Exams
  \caltext{5/15/2017}{\color{blue}Entrega I}
  \caltext{5/29/2017}{\color{blue}Entrega II}
  \caltext{7/3/2017}{\color{blue}Entega III}
  \caltext{7/19/2017}{\color{blue}Presetación final e informe}

\end{calendar}

\section*{Aspectos Administrativos}
\begin{itemize}
  \item La ausencia a controles y el atraso en la entrega de trabajos debe ser informada y justificada a la coordinación del Magister (mpgi@ing.puc.cl), con anterioridad a la fecha y hora de entrega. De lo contrario, se descontarán puntos de acuerdo al tipo de trabajo y días de atraso.

  \item Se considerará la asistencia y puntualidad en la llegada a clases como parte de la aprobación del curso.
\end{itemize}

\section*{Código de Honor de todo estudiante de la Pontificia Universidad Católica de Chile}

Los alumnos de la Escuela de Ingeniería de la Pontificia Universidad
Católica de Chile (y alumnos de los cursos dictados por esta
Escuela) deben mantener un comportamiento acorde a la Declaración de
Principios de la Universidad. En particular, se espera que mantengan
altos estándares de honestidad académica.  Cualquier acto
deshonesto o fraude académico está prohibido; los alumnos que
incurran en este tipo de acciones se exponen a un Procedimiento
Sumario. Es responsabilidad de cada alumno conocer y respetar el
documento sobre Integridad Académica publicado por la Dirección de
Docencia de la Escuela de Ingeniería.

\end{document}
