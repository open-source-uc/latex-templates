% -------------------------------------------------------------
% Hoja de Fórmulas
% -------------------------------------------------------------
\headingFormulaSheet%
\subsection*{Formulas Esenciales}%
%
\[ f'(x)=\lim_{h \to 0}\frac{f(x+h)-f(x)}{h}\]
\[ f'(c)=\lim_{b\to a}\frac{f(b)-f(a)}{b-a}\]
\[(f(x)\pm g(x))'=f'(x)\pm g'(x) \]
\[ (f(x) \cdot g(x))'=f'(x) \cdot g(x) + f(x) \cdot g'(x)\]
\[ \left(\frac{f(x)}{g(x)}\right)'= \frac{f'(x)\cdot g(x)-f(x)\cdot g'(x)}{g^{2}(x)}
\]
\[\left(x^{n} \right)'= n\cdot x^{n-1}\]
\[ \left(f(g(x)\right)'=f'(g(x))\cdot g'(x)\]
\[\left(f^{-1}(x)\right)'=\frac{1}{f'(f^{-1}(x))}\]

\subsection*{Derivadas Notables}
\[ (\sin(x))'=\cos(x) \]
\[(\cos(x))'=-\sin(x) \]
\[ (\tan(x))'=\sec^{2}(x) \]
\[(\sec(x))'=\sec(x)\tan(x)\]
\[ (\csc(x))'=-\csc(x)\cot(x)\]
\[ (\cot(x))'=-\csc^{2}(x) \]
\[ (\sin^{-1}(x))'=(\arcsin(x))'=\frac{1}{\sqrt{1-x^{2}}} \]
\[(\cos^{-1}(x))'=(\arccos(x))'=-\frac{1}{\sqrt{1-x^{2}}}\]
\[(\tan^{-1}(x))'=(\arctan(x))'=\frac{1}{1+x^{2}}\]
\[(\sec^{-1}(x))'=(\arcsec(x))'=\frac{1}{\abs{x} \sqrt{x^2-1}}\]
\[(\csc^{-1}(x))'=(\arccsc(x))'=-\frac{1}{\abs{x}\ \sqrt{x^2-1}}\]
\[(\cot^{-1}(x))'=(\arccot(x))'=-\frac{1}{1+x^{2}}\]
\[(a^{x})'=a^{x} \cdot \ln(a) \\ a\in \mathbb{R}\]
\[(e^{x})' = e^{x}\]
\[(\log_{a}{b})'=\frac{1}{\ln{a} \cdot b}\]

% --------------------------------------------
\subsection*{Análisis gráfico}
$f'(x)>0, \forall x\in \mathopen]a,b\mathclose[ \Rightarrow f(x)$ es creciente en $[a,b]$
\[f'(x)<0 ,\\ \forall x\in \mathopen]a,b\mathclose[\\ \Rightarrow f(x)\\ \mbox{ es decreciente en $[a,b]$}  \]
\[f''(x)>0, \forall x\in \mathopen]a,b\mathclose[ \\ \Rightarrow f(x) \mbox{ es concava hacia arriba en $[a,b]$}\]
\[f''(x)<0, \forall x\in \mathopen]a,b\mathclose[ \\ \Rightarrow f(x) \mbox{ es concava hacia abajo en $[a,b]$} \]
\textbf{Punto Crítico:} todo $x$ tal que $f'(x)=0$\\
\textbf{Punto de Inflexión:} todo $x$ tal que $f''(x)=0$

\subsection*{Máximos y Mínimos}
Los máximos y mínimos locales de una función $f(x)$ son aquellos puntos donde:
\begin{enumerate}
	\item Hay un máximo o mínimo absoluto.
	\item Son los puntos críticos de la función.
\end{enumerate}

Para verificar cuales son máximos o mínimos se debe ver lo siguiente:
\begin{itemize}
	\item Si a la izquierda del punto es creciente y a la derecha es decreciente, el punto es máximo local. Por otro lado, si a la izquierda del punto es decreciente y a la derecha es creciente, el punto es mínimo local.
	\item Si $f''(x)>0$, x es mínimo local, y si $f''(x)<0$, x es máximo local.
	\item Para verificar cuál es el valor máximo absoluto o mínimo absoluto hay que evaluar el punto en la función.
\end{itemize}
%
\subsection*{Nueva Asíntota, Asíntota Oblicua}
%
Se define como asíntota oblicua de la curva $y=f(x)$ como aquella recta $y=mx+n$ que cumple las siguientes propiedades:\\
$$ m = \lim_{x \to \pm\infty}\frac{f(x)}{x}$$
$$ n = \lim_{x \to \pm\infty} f(x)-xm$$
La asíntota oblicua de la función $f(x)$, si existe, es aquella recta de la siguiente forma:
\[y=\lim_{x \to \pm\infty}\frac{f(x)}{x}\cdot x + \lim_{x \to \pm\infty} f(x)-xm\]
\textbf{Observación:} La máxima cantidad de asíntotas oblicuas y asíntotas horizontales es 2
%
\subsection*{Regla de L'Hopital (Hecha por Bernoulli)}

La regla de L'Hopital se usa para calcular límites de funciones que al calcularles el límite queda de la forma $\frac{0}{0}$, $\frac{\infty}{\infty}$, $0 \cdot \infty$, $\infty-\infty$, $0^{0}$, $1^{\infty}$, $\infty^{\infty}$ y $\infty^{0}$. Para aplicar la Regla de L'Hopital se deben transformar estas formas a $\frac{0}{0}$ o $\frac{\infty}{\infty}$. Una vez transformada la  función entonces se puede aplicar la Regla de L'Hopital, la cual es la siguiente:
\[\lim_{x\to a} \frac{f(x)}{g(x)}=\lim_{x\to a} \frac{f'(x)}{g'(x)}\]