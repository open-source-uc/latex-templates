\documentclass[addpoints,fleqn,11pt]{exam}

% -----------------------------------------------------
% EDITAR - Información del Curso y Prueba
% -----------------------------------------------------
\newcommand{\codigoDelCurso}{MAT1610}                 % EDITAR - Código del Curso (ej. MAT1203)
\newcommand{\elCurso}{Cálculo I}                      % EDITAR - Título del Curso (ej. Cálculo I)
\newcommand{\fecha}{Mayo 2020}                        % EDITAR - Fecha del Documento (ej. Mayo 2020)
\newcommand{\tipoDocumento}{estudiatón}               % EDITAR, tipo del documento. (Puede ser interrogación, control, estudiatón o guía)
\newcommand{\nombreDocumento}{Estudiatón I1}          % EDITAR, tipo del documento. (Libre, ej. Interrogación 1 o Guía de Estudio)
\newcommand{\libroAC}{Cerrado}                        % EDITAR - Tipo de Examen (Cerrado o Abierto)
\newcommand{\conCalculadoras}{No}                     % EDITAR - ¿Se permiten calculadoras? (Si o no)
\newcommand{\hojaDeFormulas}{Si}                      % EDITAR - ¿Incluir hoja de fórmulas? (Si o no)
\newcommand{\tieneSecciones}{No}                      % EDITAR - ¿El documento tiene secciones? (Si o no)
                                                      %        If Yes: Use \sectionStart to start new sections
                                                      %        If No: Comment all \sectionStart lines
\newcommand{\respuestaPorSeparado}{Si}                % EDITAR - Se responde las preguntas en un libro separado? (Si o no)
\newcommand{\preguntasPorResponder}{3}                % EDITAR - Número de preguntas a responder del total, (ej. Todas o 3)
\newcommand{\conRespuestas}{Si}                       % EDITAR - Incluir soluciones (Si o no)
\newcommand{\soportaEmojis}{Si}                       % EDITAR - ¿El documento soporta emojis? (Si o no, requiere LuaLaTeX-dev)

% -----------------------------------------------------
% INICIO DEL DOCUMENTO
% -----------------------------------------------------

% -------------------------------------------------------------
% Paquetes
% -------------------------------------------------------------
\usepackage{microtype}
\usepackage{commath}
% Configura el espacio tipográfico por razones estéticas
\usepackage{cmbright}                                 % Tipografía (Computer Modern Bright)
\usepackage[colorlinks = true,
    allcolors = black]{hyperref}              % Hiper-referencias en el PDF
\usepackage{subfigure}    % Entorno de multiples figuras
\usepackage{amsmath}   % Matemáticas estándar
\DeclareMathOperator{\arcsec}{arcsec}
\DeclareMathOperator{\arccot}{arccot}
\DeclareMathOperator{\arccsc}{arccsc}
\usepackage{amssymb}                                  % Símbolos matemáticos estándar
\usepackage{graphicx}                                 % Para incluir imágenes
\usepackage{booktabs}                                 % Tablas profesionales
\usepackage[spanish]{babel}                           % Definición del lenguaje

% Deshabilitado: Tamaño A4
% \usepackage[a4paper,                          % Tamaño del papel
%     top = 25mm,                               % Margen superior
%     bottom = 25mm,                            % Margen inferior
%     left = 25mm,                              % Margen izquierdo
%     right = 25mm]{geometry}                   % Margen derecho

\usepackage[dvipsnames]{xcolor}                       % Colores avanzados
\usepackage{soul}                                     % Resaltar texto y otras cosas
\usepackage{caption}                                  % Descripciones en objetos flotantes
%\usepackage{textcomp}                                 % Símbolo de euro (obsoletar?)
\usepackage{ifthen}                                   % Condicionales if/then
\usepackage{fmtcount}                                 % Convertir enteros a palabras
\usepackage[per-mode = symbol]{siunitx}	              % Unidades del S.I.U
\usepackage[version=4]{mhchem}                        % Símbolos químicos
\usepackage{totcount}                                 % Mostrar contador de puntaje al princpio
\usepackage{xstring}                                  % Para manipular strings
\usepackage{multicol}                                 % Entorno de multiples columnas (FS)
\usepackage{lipsum}                                   % Texto Lorem Ipsum (obsoletar?)

\usepackage[no-math]{fontspec}
\newfontfamily{\emoji}{emoji.ttf}[Renderer=Harfbuzz] % "Enunciados con emojis" - ΠCAi
\setsansfont{cmunb}[        % CM Bright para Texto
    Extension=.otf,
    UprightFont=*mr, % normal
    ItalicFont=*mo, % itálica
    BoldFont=*sr, % seminegrita
    BoldItalicFont=*so, % seminegrita itálica
]

\usepackage{wasysym}                                  % Símbolos para preguntas de alternativas

% -----------------------------------------------------
% Definiciones del Documento
% -----------------------------------------------------
\pointsinrightmargin{}                                  % Puntos en el margen derecho
\pointformat{\bfseries(\themarginpoints)}  % PUNTOS, agregar condicional!
% Puntos en negrita y formato de paréntesis
\pointformat{}
\renewcommand{\partlabel}{\bfseries{(\thepartno)}}    % Números de parte negrita
\renewcommand{\subpartlabel}{\bfseries{\thesubpart{}}} % Números de parte negrita
\pagestyle{headandfoot}                               % Header y Footer de la solución
\header{}{}{}                                         % Header
\footer{}{}{\theCourseCode{}                          % Footer
    \theCourse{} --- \theExamDate{}
    (Estudiatón)}
\graphicspath{{imagenes/}}                              % Directorio con las imágenes
\definecolor{myRed}{RGB}{184,0,1}                     % Nuevo gris
\definecolor{myGrey}{RGB}{211,211,211}                % Nuevo rojo
\checkedchar{\large \(\CIRCLE\)}
\setlength{\fboxrule}{0pt}

% -----------------------------------------------------
% Versiones
% -----------------------------------------------------
\newcommand{\theVersion}{1.3}
\newcommand{\theVersionDate}{24-May-2020}

% -----------------------------------------------------
% Nuevos Comandos
% -----------------------------------------------------
\newtotcounter{pointsCounter}                         % Contador del número de puntos por parte
\newtotcounter{numberOfSections}                      % Contador de número de secciones
\newtotcounter{numberOfQuestions}                     % Contador de número de preguntas

% -----------------------------------------------------
% Estructura del Documento - Preguntas y secciones
% -----------------------------------------------------
\ifthenelse{
    \equal{\hasSections}{Yes} \OR{} \equal{\hasSections}{Y} \OR{} \equal{\hasSections}{yes} \OR{} \equal{\hasSections}{y} \OR{} \equal{\hasSections}{YES}
}
% Si es que tiene secciones
{\newcommand{\sectionStart}{                           % Initiate new sections
        \setcounter{question}{0}                          % (Re)initiate questions counter
        \setcounter{figure}{0}                            % Initiate figure counter
        \setcounter{table}{0}                             % Initiate table counter
        \qformat{\ifthenelse{\equal{\thequestion}{1}}
            {\fbox{\parbox{16cm}{\textbf{\large SECCIÓN \Alph{numberOfSections} \hfill\vrule depth 0.5em width 0pt \\
                            Question \Alph{numberOfSections}\thequestion}} \hfill\vrule depth 1.5em width 0pt}}
            {\fbox{\parbox{16cm}{\large \textbf{Pregunta \Alph{numberOfSections}\thequestion}} \hfill\vrule depth 0.8em width 0pt}}
        }
        \stepcounter{numberOfSections}                    % Incrementar contador
    }}
% Si no hay secciones
{   \setcounter{question}{0}                          % Reinicializar contador de preguntas
    \setcounter{figure}{0}                            % Iniciar contador de figuras
    \setcounter{table}{0}                             % Iniciar contador de tablas
    \qformat{
        {\fbox{\parbox{16cm}{\large \textbf{Pregunta \thequestion}} \hfill\vrule depth 0.8em width 0pt}}
    }
}

\newcommand{\callQuestion}[2]{                        % Nueva pregunta
    \cleardoublepage{} \input{#1}                       % Iniciar pregunta en nueva página
    \addtocounter{pointsCounter}{#2}                  % Incrementar puntos
    \addtocounter{numberOfQuestions}{1}               % Incrementar preguntas
    \stepcounter{figure}                              % Incrementar figuras
    \stepcounter{table}                               % Incrementar tablas
}

% Define la hoja de fórmulas
\newcommand{\callFormulaSheet}[1]{
    \ifthenelse{
        \equal{\formulaSheet}{Yes} \OR{} \equal{\formulaSheet}{Y} \OR{} \equal{\formulaSheet}{yes} \OR{} \equal{\formulaSheet}{y} \OR{} \equal{\formulaSheet}{YES}}{\cleardoublepage{} \input{#1}}{}
}

\newcommand{\headingFormulaSheet}{
    \pagestyle{empty}
    \setlength{\textwidth}{180mm}
    \setlength{\textheight}{265mm}
    \setlength{\topmargin}{-20mm}
    \setlength\oddsidemargin{-12mm}
    \setlength\evensidemargin{-12mm}
    \setlength{\columnsep}{10mm}
    \setlength{\columnseprule}{0.4pt}
    \setlength{\jot}{5pt}
    \setlength\parindent{0pt}
    %
    \twocolumn
    \fcolorbox{white}{myGrey}{\begin{minipage}{20em}
            \centering \vspace{4mm} {\Large \textbf{Hoja de Fórmulas}} \\
            \theCourse\ (\theCourseCode) \\
            {\small \theExamDate{} (\theDiet)} \vspace{3mm}
        \end{minipage}}
}

% -----------------------------------------------------
% Títulos y formatos
% -----------------------------------------------------

%\qformat{\textbf{\large Question \thequestion} \hfill\vrule depth 1.5em width 0pt}

% Formato del título de la pregunta - solo las numeradas y no en partes
\pointname{}                                          % No title for marks
\pointsdroppedatright{}                               % Puntaje por pregunta
\framedsolutions{}                                      % Marco para soluciones
\addto\captionsenglish{                               % Formato de etiquetas de figura
    %    \renewcommand\figurename{Figure \thequestion .}}
    \renewcommand\figurename{Figura}}
\addto\captionsenglish{                               % Etiquetas de tabla
    %    \renewcommand\tablename{Table \thequestion .}}
    \renewcommand\tablename{Tabla}}
\makeatletter
\def\fnum@figure{\figurename~\Alph{numberOfSections}\thefigure}
\def\fnum@table{\tablename~\Alph{numberOfSections}\thetable}
\makeatother
\renewcommand{\solutiontitle}{
    \noindent\textbf{Solución:}\par\noindent}  % Renombrar soluciones
                                          % Definiciones de estilo y estructura
\begin{document}                                      % Inicio del contenido

% -----------------------------------------------------
% Portada del Documento (automatizada, revisar documento.tex)
% -----------------------------------------------------
\begin{coverpages}
    \begin{flushleft}

        \includegraphics[height=45mm]{puc_comunitario.png}
        \includegraphics[height=45mm]{delegados2020.png} \\ \vspace{10mm}

        \textcolor{MidnightBlue}{\textbf{\Large \codigoDelCurso{}: \elCurso{} --- \fecha{} (\nombreDocumento)}} \\[\baselineskip]

        \descripcionPortada{}

        \siEsEvaluacion{
            \textbf{Código de Honor:} \say{Como miembro de la comunidad de la Pontificia Universidad Católica de Chile, me comprometo a respetar los principios y normativas que la rigen. Asimismo, me comprometo a actuar con rectitud y honestidad en las relaciones con los demás integrantes de la comunidad y en la realización de todo trabajo, particularmente en aquellas actividades vinculadas a la docencia, al aprendizaje y la creación, difusión y transferencia del conocimiento.}\\[\baselineskip]

            \siResponderTodo{Para esta evaluación \textbf{debes intentar responder todas las preguntas.}}{Para esta evaluación, debes responder \textbf{solo} \NUMBERstringnum{\total{numeroDePreguntas}} preguntas.}

            \siConCalculadoras{El uso de calculadoras científicas autorizadas está \textbf{permitido}.}{El uso de calculadoras de cualquier tipo está \textbf{prohibido}.}%

            \siEsLibroAbierto{%
                Esta es una evaluación a \textbf{libro abierto}, por lo que puedes utilizar apuntes o libros acorde con el código de honor. El plagio se encuentra prohibido.
            }
            {
                Tienes \textbf{prohibido} el uso de cualquier clase de apuntes, libros o hojas de referencia, al menos que una sea incluida con el documento.
            }
            \siRespuestaPorSeparado
            {%
                Para responder, \textbf{debes escribir tus respuestas por separado}, en una hoja o documento que se te debería entregar. No es aceptable entregar las respuestas en esta hoja.\\[\baselineskip]
            }
            {%
                Para responder, \textbf{debes escribir tus respuestas en este documento}. No es aceptable entregar las respuestas por separado, salvo que sea explícitamente solicitado.\\[\baselineskip]
            }
        }{}

        \siHojaDeFormulas{Este documento \textbf{incluye una hoja de fórmulas} hacia el final. Puedes usarla como referencia.\\} % if

        \vfill
        % Advertencia de respuestas
        \siConRespuestas{\textcolor{myRed}{\LARGE \textbf{Advertencia:} Esta copia incluye respuestas.}}{}
        \vfill

        \textcolor{MidnightBlue}{\textbf{\large Instrucciones Adicionales}}\\

        Todos en el Zoom van a ser divididos en grupos de \textbf{5 a 8 personas} al iniciar la estudiatón.\\ Una vez que seas asignado a un grupo, \textbf{tendrás que presentarte antes de empezar} y después podrán comenzar a hacer en orden cada pregunta.\\[\baselineskip]

        \textbf{Si tienen alguna duda, primero intenten llevarlo al grupo en el que están.} Si nadie puede resolver su duda, pueden solicitar ayuda desde Zoom para que vaya un tutor a ayudar. Si todos en tu grupo terminan la guía, o están muy pegados con una pregunta de desarrollo, pueden pedir la pauta de esta guía.

        \vspace{8mm} \textcolor{MidnightBlue}{\small \unidadAcademica,  Pontificia Universidad Católica de Chile}
        %
    \end{flushleft}
\end{coverpages}
\newpage

% Impresión automática de respuestas, NO REMOVER %
\siConRespuestas{\printanswers}{}                                   % Portada del Documento
\begin{questions}                                     % Inicio de las preguntas
    % -----------------------------------------------------
    % Iniciar preguntas
    % -----------------------------------------------------
    % EDITAR - Estructura del Documento - Secciones y preguntas
    % -----------------------------------------------------
    % EDITAR - Copiar para añadir secciones o preguntas
    % -----------------------------------------------------
    \traerPregunta{preguntas/alternativas.tex}{15}    % EDITAR - Archivo de la pregunta y su puntaje
    \traerPregunta{preguntas/doce.tex}{20}
    \traerPregunta{preguntas/trece.tex}{20}
    \traerPregunta{preguntas/catorce.tex}{20}
    % -----------------------------------------------------
    % Terminar preguntas
    % -----------------------------------------------------

\end{questions}                                       % Finalizar secciones y preguntas
\vfill \medskip\hfill\textbf{FIN DE LA GUÍA}          % Finalizar documento principal

\traerHojaDeFormulas{formulas.tex}                    % EDITAR - Hoja de fórmulas, comentar línea si no hay una.

% -----------------------------------------------------
% Finalizar documento
% -----------------------------------------------------
\end{document}
