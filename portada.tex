
% -----------------------------------------------------
% Portada del Documento
% -----------------------------------------------------
\begin{coverpages}
    \begin{flushleft}

        \includegraphics[height=45mm]{puc_comunitario.png}
        \includegraphics[height=45mm]{delegados2020.png} \\ \vspace{10mm}

        \textcolor{MidnightBlue}{\textbf{\Large \theCourseCode{} \theCourse{} --- \theExamDate{} (Estudiatón I2)}} \\[\baselineskip]
        \vspace{10mm}
        Esta guía fue hecha con {\emoji \symbol{"2764"}} por novatos para novatos, para la estudiatón de cálculo el 13 de mayo de 2020, específicamente en preparación a la I2. Incluso si no pudiste estar en la estudiatón, esperamos que esta guía te sea útil para practicar y prepararte.\\[\baselineskip]

        Esta guía tiene 14 preguntas, 11 de alternativas y 3 de desarrollo. Estimamos el tiempo completo en 1h y 40m, sin contar el desafío (último ejercicio).\\[\baselineskip]

        Puedes encontrar una hoja de fórmulas en la penúltima página de este documento, aunque te \textbf{recomendamos enfáticamente no usarla} al menos que te encuentres pegado en un ejercicio.

        \vfill
        %
        \textcolor{MidnightBlue}{\textbf{\large Notas para Participantes}}\\

        Todos en el Zoom van a ser divididos en grupos de \textbf{5 a 8 personas} al iniciar la estudiatón.\\ Una vez que seas asignado a un grupo, \textbf{tendrás que presentarte antes de empezar} y después podrán comenzar a hacer en orden cada pregunta.\\[\baselineskip]

        \textbf{Si tienen alguna duda, primero intenten llevarlo al grupo en el que están.} Si nadie puede resolver su duda, pueden solicitar ayuda desde Zoom para que vaya un tutor a ayudar. Si todos en tu grupo terminan la guía, o están muy pegados con una pregunta de desarrollo, pueden pedir la pauta de esta guía.

        \vspace{3mm} \textcolor{MidnightBlue}{\small Delegados Generacionales 2020 Ing. UC}
        %
    \end{flushleft}
\end{coverpages}
\newpage
%
\ifthenelse{
    \equal{\withSolutions}{Yes} \OR{} \equal{\withSolutions}{Y} \OR{} \equal{\withSolutions}{yes} \OR{} \equal{\withSolutions}{y} \OR{} \equal{\withSolutions}{YES}}{\printanswers}{}