
% -----------------------------------------------------
% Portada del Documento (automatizada, revisar documento.tex)
% -----------------------------------------------------
\begin{coverpages}
    \begin{flushleft}

        \includegraphics[height=45mm]{puc_comunitario.png}
        \includegraphics[height=45mm]{delegados2020.png} \\ \vspace{10mm}

        \textcolor{MidnightBlue}{\textbf{\Large \codigoDelCurso{}: \elCurso{} --- \fecha{} (\nombreDocumento)}} \\[\baselineskip]

        \descripcionPortada{}

        \siEsEvaluacion{
            \textbf{Código de Honor:} \say{Como miembro de la comunidad de la Pontificia Universidad Católica de Chile, me comprometo a respetar los principios y normativas que la rigen. Asimismo, me comprometo a actuar con rectitud y honestidad en las relaciones con los demás integrantes de la comunidad y en la realización de todo trabajo, particularmente en aquellas actividades vinculadas a la docencia, al aprendizaje y la creación, difusión y transferencia del conocimiento.}\\[\baselineskip]

            \siResponderTodo{Para esta evaluación \textbf{debes intentar responder todas las preguntas.}}{Para esta evaluación, debes responder \textbf{solo} \NUMBERstringnum{\total{numeroDePreguntas}} preguntas.}

            \siConCalculadoras{El uso de calculadoras científicas autorizadas está \textbf{permitido}.}{El uso de calculadoras de cualquier tipo está \textbf{prohibido}.}%

            \siEsLibroAbierto{%
                Esta es una evaluación a \textbf{libro abierto}, por lo que puedes utilizar apuntes o libros acorde con el código de honor. El plagio se encuentra prohibido.
            }
            {
                Tienes \textbf{prohibido} el uso de cualquier clase de apuntes, libros o hojas de referencia, al menos que una sea incluida con el documento.
            }
            \siRespuestaPorSeparado
            {%
                Para responder, \textbf{debes escribir tus respuestas por separado}, en una hoja o documento que se te debería entregar. No es aceptable entregar las respuestas en esta hoja.\\[\baselineskip]
            }
            {%
                Para responder, \textbf{debes escribir tus respuestas en este documento}. No es aceptable entregar las respuestas por separado, salvo que sea explícitamente solicitado.\\[\baselineskip]
            }
        }{}

        \siHojaDeFormulas{Este documento \textbf{incluye una hoja de fórmulas} hacia el final. Puedes usarla como referencia.\\} % if

        \vfill
        % Advertencia de respuestas
        \siConRespuestas{\textcolor{myRed}{\LARGE \textbf{Advertencia:} Esta copia incluye respuestas.}}{}
        \vfill

        \textcolor{MidnightBlue}{\textbf{\large Instrucciones Adicionales}}\\

        Todos en el Zoom van a ser divididos en grupos de \textbf{5 a 8 personas} al iniciar la estudiatón.\\ Una vez que seas asignado a un grupo, \textbf{tendrás que presentarte antes de empezar} y después podrán comenzar a hacer en orden cada pregunta.\\[\baselineskip]

        \textbf{Si tienen alguna duda, primero intenten llevarlo al grupo en el que están.} Si nadie puede resolver su duda, pueden solicitar ayuda desde Zoom para que vaya un tutor a ayudar. Si todos en tu grupo terminan la guía, o están muy pegados con una pregunta de desarrollo, pueden pedir la pauta de esta guía.

        \vspace{8mm} \textcolor{MidnightBlue}{\small \unidadAcademica,  Pontificia Universidad Católica de Chile}
        %
    \end{flushleft}
\end{coverpages}
\newpage

% Impresión automática de respuestas, NO REMOVER %
\siConRespuestas{\printanswers}{}