% -----------------------------------------------------
% QUESTION
% -----------------------------------------------------
\question
\begin{figure}[!ht]
   \centering
   \includegraphics[width=0.8\textwidth]{gprima.png}
%   \includegraphics[height=60mm]{RandomImage.jpg}
   \caption{Función $g'(x)$.}
   \label{fig:questionA1}                             % EDIT - Create unique labels for figures
   \addtocounter{figure}{-1}
\end{figure}
En base a la figura~\Alph{numberOfSections}\ref{fig:questionA1} describiendo $g'(x)$, analice la función y responda las siguientes preguntas.

\begin{parts}
% -----------------------------------------------------
\part Determine dónde la derivada de $g(x)$ se anula.
\droppoints

\begin{solution}
Dado que se tiene el gráfico de la derivada ($g'(x)$), solo necesitamos buscar donde la ordenada es $0$, que en este caso es $x = 4$ y $x = 0$.
% -----------------------------------------------------
% To include a handwritten solution, scan or photograph
% it and save it in the Images folder. Then use the
% following code to include it in the exam paper,
% simply change the name of the image file within the
% \includegraphics LaTeX command.
% -----------------------------------------------------
\end{solution}

% -----------------------------------------------------
\part[5] Determine los intervalos de crecimiento y decrecimiento de $g(x)$.
\droppoints

\begin{solution}
Buscaremos los intervalos en donde nuestra derivada es positiva y negativa. Tenemos que tener mucho ojo en tomar valores en el eje $x$, puesto que justo ahí el valor de nuestra derivada es $0$. Dicho esto, el intervalo de crecimiento es $(0, 4)$ y decrece en $(-3, 0) \cup (4, 5)$.
\end{solution}

% -----------------------------------------------------
\part[5] Determine la concavidad de $g(x)$ en el intervalo $(-3, 5)$.
\droppoints

\begin{solution}
Para calcular la concavidad, solo debemos analizar el signo de nuestra segunda derivada.

En $(-3,-2) \cup (2,5)$ nuestra funcion es cóncava hacia abajo, puesto que $G''(x) < 0$.

En el intervalo $(-2,2)$ nuestra $G''(x) > 0$, por lo que será cóncava hacia arriba.
\end{solution}

\part[5] Determine los máximos y mínimos locales de $g(x)$ en el dominio del gráfico.
\droppoints

\begin{solution}
Para determinar los puntos mínimos y maximos tenemos que analizar nuestra segunda derivada en los puntos críticos de la función, es decir, en donde nuestra primera derivada sea igual a $0$.\\

Notamos que nuestra derivada es $0$ en $x=0$ y $x=4$, por lo que nos toca analizar el signo de nuestra segunda derivada en las cercanías de $x = 0$ y $x = 4$\\

En $x=0$ nuestra G’’(x) es $>0$, por lo qué hay un mínimo local en $x=0$, $y = 0$.
\end{solution}
\end{parts}
